% Snapshot Objectives - Audio Visualizer (Dynamic Graphic Display)
% LaTeX Document
\documentclass[12pt]{article}
\usepackage{geometry}
\geometry{margin=1in}
\usepackage{enumitem}

\title{Snapshot Objectives \\ Audio Visualizer}
\author{Group 11 \\ Isaac Trejo \\ Adrian Perez-Gonzalez \\ Camrynne Paul \\ Osame Osayande}
\date{\today}

\date{\today}

\begin{document}
\maketitle

\section{Start Objective}
The initial goal is to develop the basis of our Audio Visualizer project.
\\

  An audio visualizer is a dynamic graphic display that converts sound—such as music, podcasts, voice recordings, or ambient audio—into animated visuals. The visuals can be shapes, colors, lines, particles, or text, all of which change in real time based on the properties of the audio signal, specifically frequency (pitch) and amplitude (volume).
\\

  The purpose of this project is to provide a system similar to visualizers found in media players, video editing software, streaming overlays, and music production tools. These visualizers enhance the listening experience by making audio more interactive and visually expressive.
\\

During this initial phase, our goal is to:
\begin{itemize}
    \item Build the base system for capturing or importing audio.
    \item Implement basic audio analysis (frequency bands, amplitude levels).
    \item Create initial animated graphics that react to audio input.
    \item Establish the structure of the visualization engine.
\end{itemize}

We will divide tasks between audio processing and visual rendering to ensure the system can react to sound smoothly and responsively.

\section{Checkpoint 1}
With the core framework in place, the focus of Checkpoint 1 is to expand the visualizer's capabilities and introduce more sophisticated features that enhance responsiveness and visual diversity.

Planned additions include:
\begin{itemize}
    \item Frequency Spectrum Visualization: Bars or waves that represent different frequency ranges.
    \item Amplitude-Based Animation: Shapes or colors responding to volume changes.
    \item Enhanced Color Mapping: Dynamic gradients based on audio intensity.
    \item User Controls: Options to adjust sensitivity, theme, color palette, and animation modes.
\end{itemize}

We will also begin working on a user interface that allows users to switch between visualization styles.

\section{Checkpoint 2}
For Checkpoint 2, the goal is to refine the visualization system and introduce more advanced graphic elements, making the visualizer feel more professional and expressive.

Planned features include:
\begin{itemize}
    \item Waveform Visuals: Real-time drawing of the audio waveform.
    \item Particle Systems: Motion graphics driven by frequency peaks.
    \item Text Reactive Elements: Words or captions that pulse or distort with sound.
    \item Preset Library: A selection of unique visualization styles.
\end{itemize}

During this checkpoint, we will also refine animation smoothness and overall performance so visuals react fluidly to the audio.

\section{Final Checkpoint}
In the final phase of the project, our goal is to polish the Audio Visualizer and ensure it performs efficiently across different audio sources and visualization modes.

\subsection{Front-End Improvements}
\begin{itemize}
    \item UI Enhancements: Smoother interaction and easier mode switching.
    \item Visual Polish: Better transitions, cleaner animations, improved color harmony.
    \item Accessibility Options: Adjustable contrast and simplified modes.
\end{itemize}

\subsection{Back-End Improvements}
\begin{itemize}
    \item Performance Optimization: Faster rendering and audio processing.
    \item Low-Latency Response: Ensures visuals match the audio in real time.
    \item Improved Audio Analysis: Sharper detection of peaks and frequency changes.
\end{itemize}

\section{Conclusion}
We initially considered exploring advanced features such as AI-driven beat detection and generative visual patterns, but these will be part of a future development phase. Overall, the team is satisfied with the progress of the Audio Visualizer and is excited about the potential for future expansion.

\end{document}
